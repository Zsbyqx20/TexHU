\documentclass[texList,hyper]{TexHU}

\title{Modern Computer Architecture: Final Project Design}
\titleZh{现代计算机体系架构:综合课程设计}
\author{刘国鸿}

\begin{document}
\maketitle
\section{实验介绍}\label{sec:01}

神经网络是目前最热的技术之一。神经网络加速器是为了提高神经网络算法而提出的新型计算单元。与传统的CPU、GPU相比,神经网络加速器可以大幅度提高性能、降低功耗。然而,高效的神经网络加速器通常只支持有限的网络算子,而在完整的神经网络应用中,通常还需要一些其他加速设备对数据进行预处理等其他操作。

本次大作业提供了一个简单的神经网络加速器仿真框架,通过实验研究如何进行涉及多个不同加速器的系统仿真,并尝试一些优化方案。


This is a \textbf{random} text for \textsc{testing} the \textit{performance} of such a new font.

Let's try some different ligatures, such as !!, !?, \textit{ffl}, ffi.

There is roughly a \(50\%\) 中文插入chance.

% \cfs 中文测试 testing for English.
\[a^2+b^2=c^2=\pi\sqrt[3]{a}\]

\sectionDivider
\section{实验内容}
正如\hyperref[sec:01]{这里}提到,。
\begin{texList}
    this is a piece of Python code.
\end{texList}


\begin{figure}
    \centering
    \frame{\colorbox{red}{\includegraphics[width=0.8\textwidth]{assets/test.jpg}}}
    \caption{被剪裁和遮盖的图片}
  \end{figure}
  
\end{document}